\section{Vector bundles on differentiable manifolds}
    \subsection{Quasi-coherent sheaf cohomology} \label{subsection: quasi_coherent_sheaf_cohomology}

    \subsection{de Rham cohomology and a bit of real Hodge theory}
        Let us now discuss \say{de Rham cohomology}, which is not quite a cohomology theory per se, but rather a \say{hypercohomology theory}. Regardless, it is a remarkable application of the machinery of sheaf cohomology and can be regarded as a slightly richer topological invariants of smooth manifolds than singular cohomology, which ostensibly was engineered for the study of general topological spaces. For instance, singular cohomology captures no information about the structure sheaf of a smooth manifold, which is really what distinguishes it from a plain topological space. We ought to also note that, if $X$ is a $C^k$-manifold, then de Rham cohomology will also be a somewhat finer topological invariant than sheaf cohomology in the sense of subsection \ref{subsection: quasi_coherent_sheaf_cohomology}. This is because while one can set up sheaf cohomology to work for any $0 \leq k \leq +\infty$, de Rham cohomology is - by construction - an invariant particular to the $C^{\infty}$ case.