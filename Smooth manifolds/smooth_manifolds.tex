\section{Differentiable manifolds}
    \subsection{Topological manifolds}
        We begin by discussing topological manifolds, which are very special topological spaces which locally resemble $\R^n$ in a precise sense. For readers familiar with ringed spaces, these topological manifolds will be the underlying spaces of the smooth manifolds that we will discuss in the next subsection: to us, smooth manifolds will be topological manifolds equipped with $C^{\infty}$ structure sheaves of $\R$-algebras, so-called \say{smooth structures}.
    
        \begin{definition}[Krull dimension] \label{def: krull_dimension}
            (Cf. \cite[\href{https://stacks.math.columbia.edu/tag/0055}{Tag 0055}]{stacks}) Let $X$ be a topological space. Its \textbf{Krull dimension} is the element of $\N \cup \{+\infty\}$ given by:
                $$\dim X := \sup \{ \text{lengths of finite chains $Z_0 \subset Z_1 \subset ... \subset Z_n \subset X$, where each $Z_i$ is irreducible} \}$$
            Furthermore, we take:
                $$\dim \varnothing := 0$$
        \end{definition}
        \begin{definition}[Topological manifolds] \label{def: topological_manifolds}
            For us, the word \textbf{topological manifold} of dimension $n$ (for some $n \in \N$) (or \textbf{manifold} of dimension $n$ or \textbf{$n$-manifold}, for short) shall mean a topological space $X$ that:
            \begin{itemize}
                \item dimension $n$,
                \item is Hausdorff,
                \item admits a countable basis for its topology (i.e. \say{second-countable}), and
                \item is locally homeomorphic to some open subset of $\R^n$ with the usual Euclidean metric topology (i.e. \say{locally Euclidean}), which is to say that for each point $x \in X$, there exists an open neighbourhood $U \ni x$ and an open subset $U' \subseteq \R^n$ for which there is a homeomorphism $U \cong U'$; we note also that $n$ remains constant as $x \in X$ varies.
            \end{itemize}
            
            Sometimes, people also say \say{real manifolds}, but in our opinion, this is only a useful terminology for differentiating between \say{smooth real manifolds} (to be defined) and so-called \say{holomorphic complex manifolds}. Otherwise, the term can cause unnecessary confusion.
        \end{definition}
        \begin{remark}
            An immediate consequence of the last condition is that the connected components of a manifold of dimension $n$ are all equally of Krull dimension $n$.
        \end{remark}
        \begin{remark}[Open subsets of topological manifolds] \label{remark: open_subsets_of_topological_manifolds}
            It is easy to see that any open subset of a topological $n$-manifold is once more a topological $n$-manifold.
        \end{remark}
        \begin{example}[Punctured affine spaces as topological manifolds] \label{example: punctured_affine_spaces_as_topological_manifolds}
            Fix a natural number $n \in \N$ and then consider the \textbf{punctured real affine $n$-space} $\R^{n + 1} \setminus \{0\}$ with the subspace topology. Because $\R^{n + 1}$ is a non-singleton Hausdorff topological space, so its points are a closed and not open, so $\R^{n + 1} \setminus \{0\}$ is an open subspace of $\R^{n + 1}$. Thus, we can conclude that $\R^{n + 1} \setminus \{0\}$ is a topological $n$-manifold, according to remark \ref{remark: open_subsets_of_topological_manifolds}.
        \end{example}
        
        Of course, since the Euclidean topology on $\R^n$ is generated by open $n$-balls, it is sufficient to only require that points of a manifold lie inside open neighbourhoods that are homeomorphic to some open $n$-ball.
        \begin{convention}
            If $x_0 \in \R^n$ is a point that we designate to be a centre and if we choose some $\e > 0$, then we will write:
                $$\B^n_\e(x_0) := \{ x \in \R^n \mid \| x - x_0 \| < \e \}$$
            for the open $n$-ball of radius $r$ centered at $x_0$. Again, it is well-known that such balls form a basis for the standard Euclidean topology on $\R^n$.

            Sometimes, we will also need closed $n$-discs/balls, which shall be given by:
                $$\bbD^n_\e(x_0) := \{ x \in \R^n \mid \| x - x_0 \| \leq \e \}$$
        \end{convention}
        \begin{remark}[Some basic topological properties of manifolds]
            Definition \ref{def: topological_manifolds} also hints at the following topological properties of manifolds. We leave proofs to the reader as exercises. For what follows, let $X$ be a manifold.
            \begin{enumerate}
                \item \textbf{(Path-connectedness):} Any connected component of $X$ is path-connected. Hence, a manifold is path-connected if and only if it is connected, and any manifold is locally path-connected.
                \item \textbf{(Countably many connected components):} Each connected component of $X$ is open, and hence the collection of connected components form an open cover of $X$. Due to second-countability, this also implies that any manifold has countably many connected components.
                \item \textbf{(Para-compactness):} Any manifold is locally compact. Along with second-countability, this implies that every manifold is para-compact (every open cover has a locally finite refinement).
                \item \textbf{(Fundamental groups):} $\pi_1(X)$ is countable.
            \end{enumerate}
        \end{remark}

        Let us also remark right away that while it may seem intuitively true that if $m \not = n$ then a manifold of dimension $m$ can not be homeomorphic to one that is of dimension $n$, but there are some subtleties. We will return to this question later.

        \begin{definition}[Atlases and coordinate charts] \label{def: atlases_and_charts}
            Let $X$ be a manifold of dimension $n$ and choose a covering $X := \bigcup_{i \in I} U_i$ by open subsets $U_i$. An \textbf{atlas} for $X$ is then a family of homeomorphisms $\{ \varphi_i: U_i \xrightarrow[]{\cong} \R^n \}_{i \in I}$. Elements of an atlas are called \textbf{coordinate charts} and are usually denoted, e.g. by $(U_i, \varphi_i)$.
        \end{definition}
        
        \begin{example}[Graphs of continuous functions] \label{example: graph_coordinates}
            Let $X$ be an $n$-manifold and $U$ be an open subset thereof. If $f: U \to \R^m$ is a continuous function then let us identify its graph with the subspace $U \x \im f$ of $\R^{n + m}$. This gives rise to an open immersion:
                $$\Gamma_f: U \x \im f \to \R^n \x \R^m$$
            such that:
                $$\pr_1 \circ \Gamma_f = f \circ \pr_1|_{U \x \im f}$$
            Then, note that $\pr_1 \circ \Gamma_f$ is in fact a homeomorphism with continuous inverse given by $( \pr_1 \circ \Gamma_f )^{-1}(x) := ( x, f(x) )$ for all $x \in U$. As such, we have found a homeomorphism:
                $$\varphi \circ (\pr_1 \circ \Gamma_f): U \x \im f \xrightarrow[]{\cong} \R^n$$
            where on the RHS we have made an identification $\varphi: U \xrightarrow[]{\cong} \R^n$ (which exists, as $U$ is an open subset of an $n$-manifold), and hence have found a chart $(U \x \im f, \varphi)$.
        \end{example}
        \begin{example}[Spheres]
            An instance of the kind of coordinate charts discussed in example \ref{example: graph_coordinates} is the following.

            Let $X := \bbS^n$ be the real unit $n$-sphere, given by:
                $$\bbS^n := \{ z \in \R^{n + 1} \mid \|z\| = 1 \}$$
            and as such is a closed subset of $\R^{n + 1}$, and hence is Hausdorff and second-countable. We wish to show that $\bbS^n$ is actually an $n$-manifold, and the only thing left to check is that it is locally homeomorphic to $\R^n$.
            
            To this end, consider the function:
                $$f: \bbS^n \to \R$$
            given locally around each point $z \in \bbS^n$ by:
                $$f(x) := \sqrt{1 - \|x\|^2}$$
            for all $x \in \B^n_1(z)$. This function is clearly continuous, and thus gives rise to open immersions:
                $$\Gamma_{f, z}: \B^n_1(z) \x \im f \to \R^n \x \R$$
            for each $z \in \bbS^n$, which in turn gives rise to homeomorphisms:
                $$\varphi_z \circ (\pr_1 \circ \Gamma_{f, z}): \B^n_1(z) \x \im f \xrightarrow[]{\cong} \R^n$$
            wheren we have identified $\varphi_z: \B^n_1(z) \xrightarrow[]{\cong} \R^n$ (cf. example \ref{example: graph_coordinates}). Thus, we have shown that $\bbS^n$ is locally isomorphic to $\R^n$, and hence an $n$-manifold.
        \end{example}

        \textit{We insist on the point of view that topology and geometry are best understood structurally, which means that every construction should form some kind of a category, and we will pursue this viewpoint by investigating the kinds of limits and colimits that can exist in these categories.} Manifolds are no exception. 
        \begin{proposition}[The category of topological manifolds] \label{prop: category_of_topological_manifolds}
            \begin{enumerate}
                \item There is a category $\Mfd$ whose objects are manifolds (of all dimensions $n \in \N$) and whose morphisms are continuous maps between them.
                \item This category is closed under open immersions and finite pullbacks. Furthermore, if $X$ is an $n$-manifold and $U \hookrightarrow  X$ is an open immersion, then $\dim U = \dim X$, and the pullback of two open immersions, say $U \hookrightarrow  X, U' \hookrightarrow  X$ coincides with the intersection $U \cap U'$ (with the canonically induced open immersion $U \cap U' \to X$). 
                \item $\Mfd$ is also closed under locally closed immersions, which is to say that if $X$ is a manifold and $Z \hookrightarrow X$ is a locally closed immersion then $Z$ will also be a manifold. However, if this locally closed immersion is not a homeomorphism, then $\dim Z < \dim X$ strictly.
            \end{enumerate}
        \end{proposition}
            \begin{proof}
                \begin{enumerate}
                    \item To check that $\Mfd$ is a well-defined category, the only non-trivial thing to check is whether given manifolds $X, Y$ and a continuous map $f: X \to Y$, as well as an arbitrary open subset $V \subseteq Y$ that is homeomorphic to some $\R^m$, then its preimage $f^{-1}(V)$ will also be homeomorphic to some $\R^n$. As stated above, we can assume without any loss of generality that $V$ is homeomorphic to some open ball inside $\R^m$. Then, its preimage $f^{-1}(V)$ will be homeomorphic to some open ball inside $\R^n$, per the continuity of $f$. Any open ball in $\R^n$ is homeomorphic to $\R^n$ itself (this is well-known), so we are done.
                    \item These statements follow directly from the definition of manifolds.
                    \item Manifolds are locally compact and Hausdorff, and hence locally closed (by the Bolzano-Weierstra{\ss} Theorem). From the previous statement, we also know that if $X$ is an $n$-manifold and $U \hookrightarrow  X$ is an open immersion, then $\dim U = \dim X$. By putting the two statements together, and by assuming without any loss of generality that $X \cong \R^n$, then the assertion will following from the fact that any locally closed subset $Z \subseteq \R^n$ is an $m$-manifold for some $m \leq n$ (as they are automatically Hausdorff, second-countable, and one can prove that they are locally homeomorphic to $\R^m$ by picking charts), and that equality occurs if and only if $Z \not \cong \R^n$.
                \end{enumerate}
            \end{proof}

        The following technical lemmas are useful for constructing topological manifolds via gluing.

        \begin{lemma}[A criterion for Hausdorff-ness of quotients] \label{lemma: hausdorff_quotient_criterion}
            If $X$ is a Hausdorff toplogical space, $R$ is an equivalence relation on $X$ which is closed inside $X^2$, and if the quotient map $\pi: X \to X/R$ is open, then $X/R$ is once again Hausdorff.
        \end{lemma}
            \begin{proof}
                The quotient map $\pi: X \to X/R$ is open by assumption, so through the definition of the product topology, one gets that the \say{diagonal quotient} $\pi \x \pi: X^2 \to (X/R)^2$ is also open. $R$ is assumed to be a closed subset of $X^2$, and since $\Pi$ is an open map, $(\pi \x \pi)(X^2 \setminus R)$ has to be open inside $(X/R)^2$. We now claim that:
                    $$(\pi \x \pi)(X^2 \setminus R) = (X/R)^2 \setminus \Delta_{X/R}$$
                where $\Delta_{X/R}$ is the image of $X/R$ inside $(X/R)^2$ under the diagonal map, i.e. $\Delta_{X/R} = \{(z, z') \in (X/R)^2 \in \mid z = z'\}$; should this be the case, $\Delta_{X/R}$ would have to be closed inside $(X/R)^2$, since we have shown that $(\pi \x \pi)(X^2 \setminus R)$ is open inside $(X/R)^2$, and this would mean that $X/R$ is Hausdorff by definition. To that end, note firstly that:
                    $$(x, y) \in X^2 \setminus R \implies \pi(x) \not = \pi(y) \implies (\pi \x \pi)(x, y) \in (X/R)^2 \setminus \Delta_{X/R}$$
                which tells us that:
                    $$(\pi \x \pi)(X^2 \setminus R) \subset (X/R)^2 \setminus \Delta_{X/R}$$
                Conversely, consider:
                    $$
                        \begin{aligned}
                            (z, t) \in (X/R)^2 \setminus \Delta_{X/R} & \implies \exists (x, y) \in X^2: ( ((z, t) = (\pi \x \pi)(x, y)) \wedge ((\pi \x \pi)(x, y) \in (X/R)^2 \setminus \Delta_{X/R}) )
                            \\
                            & \implies \exists (x, y) \in X^2: ( ((z, t) = (\pi \x \pi)(x, y)) \wedge ((x, y) \not \in R) )
                            \\
                            & \implies (z, t) \in (\pi \x \pi)(X^2 \setminus R)
                        \end{aligned}
                    $$
                which tells us that:
                    $$(\pi \x \pi)(X^2 \setminus R) \supset (X/R)^2 \setminus \Delta_{X/R}$$
                Thus, we have shown that $(\pi \x \pi)(X^2 \setminus R) = (X/R)^2 \setminus \Delta_{X/R}$. As stated, this implies that $X/R$ is Hausdorff.
            \end{proof}

        \begin{lemma}[Properties of pushouts of topological spaces] \label{lemma: properties_of_pushouts_of_spaces}
            Consider a pushout diagram in $\Top$, wherein - for the sake of argument - let us suppose that $i: W \to X$ is monic, i.e. that $W$ is identified with a subspace of $X$ (otherwise, the pushout may not exist, since pushouts in $\Top$ are space attachments):
                $$
                    \begin{tikzcd}
                    W & Y \\
                    X & {X \pushout_W Y}
                    \arrow["j", from=1-1, to=1-2]
                    \arrow["i"', tail, from=1-1, to=2-1]
                    \arrow["{\iota_2}", from=1-2, to=2-2]
                    \arrow["{\iota_1}", from=2-1, to=2-2]
                    \arrow["\lrcorner"{anchor=center, pos=0.125, rotate=180}, draw=none, from=2-2, to=1-1]
                    \end{tikzcd}
                $$
            \begin{enumerate}
                \item If both $X, Y$ are locally Euclidean (not even necessarily equi-dimensional) then the pushout $X \pushout_W Y$ will also be locally Euclidean.
                \item If both $X, Y$ are second-countable then so is $X \pushout_W Y$.
            \end{enumerate}
        \end{lemma}
            \begin{proof}
                \begin{enumerate}
                    \item As we have assumed that $X$ is locally Euclidean, we can find an open cover $X := \bigcup_{a \in A} U_a$ such that each $U_a \cong \R^{n_a}$ for some $n_a \in \N$; we then have that:
                        $$X \pushout_W Y \cong ( \bigcup_{a \in A} U_a ) \pushout_W Y \cong \bigcup_{a \in A} ( U_a \pushout_{W \cap U_a} Y )$$
                    By picking an open cover $Y := \bigcup_{b \in B} V_b$ where $V_b \cong \R^{m_b}$ for some $m_b \in \N$ and arguing as above, we shall then see that:
                        $$X \pushout_W Y \cong \bigcup_{(a, b) \in A \x B} (U_a \pushout_{U_a \cap V_b} V_b)$$
                    Since the sets $U_a, V_b$ are Euclidean, so are their intersections, and hence pushouts of the form:
                        $$U_a \pushout_{U_a \cap V_b} V_b$$
                    are nothing but unions of (possibly intersecting) open subsets of $\R^{N_{a, b}}$ for some sufficiently large $N_{a, b} \in \N$. These pushouts are therefore open subsets of $\R^{N_{a, b}}$ themselves, and hence Euclidean. As the collection $\{ U_a \pushout_{U_a \cap V_b} V_b \}_{(a, b) \in A \x B}$ forms an open cover of $X \pushout_W Y$, we have thus shown that $X \pushout_W Y$ is locally Euclidean.
                    \item By choosing open covers $X := \bigcup_{a \in A} U_a$ and $Y := \bigcup_{b \in B} V_b$ where $U_a, V_b$ are basis elements, and hence the indexing sets $A, B$ are countable per the second-countability assumption on $X$ and $Y$, one easily sees that $X \pushout_W Y \cong \bigcup_{(a, b) \in A \x B} (U_a \pushout_{U_a \cap V_b} V_b)$ must also be second-countable, as a consequence of the fact that the product of two countable sets is countable.
                \end{enumerate}
            \end{proof}

        \begin{lemma}[Quotients as pushouts] \label{lemma: quotients_as_pushouts}
            Let $\C$ be any category and let:
                $$
                    \begin{tikzcd}
                    R & X & Q
                    \arrow["i", shift left=2, from=1-1, to=1-2]
                    \arrow["j"', shift right=2, from=1-1, to=1-2]
                    \arrow["\coeq", from=1-2, to=1-3]
                    \end{tikzcd}
                $$
            be a coequaliser diagram. Equivalently, this is the pushout diagram:
                $$
                    \begin{tikzcd}
                    R & X \\
                    X & Q
                    \arrow["i", from=1-1, to=1-2]
                    \arrow["j"', from=1-1, to=2-1]
                    \arrow[from=1-2, to=2-2]
                    \arrow[from=2-1, to=2-2]
                    \arrow["\lrcorner"{anchor=center, pos=0.125, rotate=180}, draw=none, from=2-2, to=1-1]
                    \end{tikzcd}
                $$
        \end{lemma}
            \begin{proof}
                Clear from the universal properties of colimits.
            \end{proof}
        \begin{remark}
            In particular, when $\C = \Top$ and when $R$ is an equivalence relation, then $Q$ will exist as the quotient $X/R$ (constructed by identifying points), meaning that the lemma can be used for constructing open coverings of quotient spaces in certain situations (e.g. when $X \cong \R^n$).
        \end{remark}
        
        \begin{example}[Real projective spaces as topological manifolds] \label{example: real_projective_spaces_as_topological_manifolds}
            Fix a natural number $n \in \N$.

            If $v, v' \in \R^{n + 1}$ are two vectors, then we shall say that they are equivalent, written $w \sim z$ if and only if there exists $\lambda \in \R^{\x}$ so that $w = \lambda z$. Equivalently, we have that:
                $$w \sim z = \R \cdot w = \R \cdot z$$
            and it is important that this is an equality, not an isomorphism, since vector spaces of the same finite dimension are automatically isomorphic. Next, let:
                $$\P^n_{\R} := (\R^{n + 1} \setminus \{0\})/\sim$$
            This construction induces a canonical quotient map:
                $$\pi: \R^{n + 1} \setminus \{0\} \to \P^n_{\R}$$
            given by:
                $$\pi(w) := \R \cdot w$$
            Let us endow $\P^n_{\R}$ with the quotient topology, with respect to which $\pi$ is continuous automatically.
            
            We claim that $\pi$ endows $\P^n_{\R}$ with the structure of a topological $n$-manifold. We will see that $\P^n_{\R}$ is actually a particularly nice manifold, and hence will find itself as a test case for many results. Due to this importance, let us prove the claim as a distinguished proposition (see proposition \ref{prop: real_projective_spaces_as_topological_manifolds}). 
        \end{example}
        \begin{lemma}
            \todo[inline]{Hausdorff criterion for group action quotients. Apply to show that $\P^n_{\R}$ is Hausdorff.}
        \end{lemma}
            \begin{proof}
                
            \end{proof}
        \begin{corollary}[Projectivisation is open] \label{coro: projectivisation_is_open}
            The quotient map:
                $$\pi: \R^{n + 1} \setminus \{0\} \to \P^n_{\R}$$
            as in example \ref{example: real_projective_spaces_as_topological_manifolds} is open.
        \end{corollary}
            \begin{proof}
                By the construction of the quotient topology, $\pi$ is continuous, so it suffices to show that $\pi^{-1}(\pi(U))$ is open in $\P^n_{\R}$ for every open subset $U \subseteq \R^{n + 1} \setminus \{0\}$.
            \end{proof}
        \begin{proposition} \label{prop: real_projective_spaces_as_topological_manifolds}
            The quotient map:
                $$\pi: \R^{n + 1} \setminus \{0\} \to \P^n_{\R}$$
            as in example \ref{example: real_projective_spaces_as_topological_manifolds} endows $\P^n_{\R}$ with a topological $n$-manifold structure.
        \end{proposition}
            \begin{proof}
                Firstly, we need to show that $\P^n_{\R}$ is Hausdorff.
                \todo[inline]{Show that $\P^n_{\R}$ is Hausdorff.}
    
                Second-countability is easy: a simultaneous application of lemmas \ref{lemma: properties_of_pushouts_of_spaces} and \ref{lemma: quotients_as_pushouts} tells us that quotients of second-countable spaces are once more second-countable. 
    
                To prove that $\P^n_{\R}$ is locally Euclidean and of dimension $n$, note firstly that when written in components, each vector $w := (w_1, ..., w_{n + 1}) \in \R^{n + 1} \setminus \{0\}$ must have at least one non-zero component, say $w_i$. If we fix $w_i$ and then denote:
                    $$[w_1 : ... : \not w_i : ... w_n] := \pi(w)$$
                then a reasonable candidate for a chart of an open neighbourhood $U_i \ni \pi(w)$ shall be:
                    $$\varphi_i: U_i \to \R^n$$
                that is given by:
                    $$
                        \begin{aligned}
                            \varphi_i( [w_1 : ... : w_{i - 1} : \not w_i : w_{i + 1} : ... w_n] ) & := \frac{1}{w_i} ( w_1, ..., w_{i - 1}, \not{w}_i, w_{i + 1}, ..., w_n )
                            \\
                            & = \left( \frac{w_1}{w_i}, ..., \frac{w_{i - 1}}{w_i}, \not{1}, \frac{w_{i + 1}}{w_i}, ..., \frac{w_n}{w_i} \right)
                        \end{aligned}
                    $$
                and this is well-defined because we are assuming that $w_i \not = 0$ and because $\left( \frac{w_1}{w_i}, ..., \frac{w_{i - 1}}{w_i}, \not{1}, \frac{w_{i + 1}}{w_i}, ..., \frac{w_n}{w_i} \right)$ obviously lies in some $n$-dimensional subspace of $\R^{n + 1}$, which is isomorphic to $\R^n$ via deletion of the constant $i^{th}$ component $1$. It is clear that $\varphi_i$ as above is bijective, with inverse $\varphi_i^{-1}: \R^n \to U_i$ is given by $\varphi_i^{-1}(x_1, ..., x_n) := (x_1, ..., x_{i - 1}, 1, x_{i + 1}, ..., x_n)$, so let us now focus on showing that both $\varphi_i$ and its inverse are continuous. To this end, let us firstly exploit the fact that $\pi: \R^{n + 1} \setminus \{0\} \to \P^n_{\R}$ is continuous to get an open subset $\tilde{U}_i := \pi^{-1}(U_i)$ of $\R^{n + 1} \setminus \{0\}$. Since $\R^{n + 1} \setminus \{0\}$ is a topological $n$-manifold, by remark \ref{remark: open_subsets_of_topological_manifolds}, such an open subset can be endowed with a chart $\tilde{\varphi}_i: \tilde{U}_i \xrightarrow[]{\cong} \R^{n + 1}$. If we now write $\pi_i: \R^{n + 1} \to \R^n$ for the (surjective) projection that deletes the $i^{th}$ entry of vectors in $\R^{n + 1}$, which is a linear map and hence is continuous, then we shall obtain a commutative diagram in $\Sets$ as follows:
                    $$
                        \begin{tikzcd}
                            {\tilde{U}_i} & {\R^{n + 1}} \\
                            {U_i} & {\R^n}
                            \arrow["{\tilde{\varphi}_i}", dashed, from=1-1, to=1-2]
                            \arrow["\pi"', from=1-1, to=2-1]
                            \arrow["{\pi_i}", from=1-2, to=2-2]
                            \arrow["{\varphi_i}", from=2-1, to=2-2]
                        \end{tikzcd}
                    $$
                Now, choose an arbitrary open subset $W \subseteq \R^n$ and then consider:
                    $$\pi( \tilde{\varphi}_i^{-1}( \pi_i^{-1}(W) ) )$$
                As $\pi_i$ is continuous, $\pi_i^{-1}(W) \subseteq \R^{n + 1}$ is open, and as $\tilde{\varphi}_i$ is a homeomorphism by virtue of being a chart, $\tilde{\varphi}_i^{-1}( \pi_i^{-1}(W) \subseteq \tilde{U}_i$ is also open; finally, because $\pi$ is an open map (see corollary \ref{coro: projectivisation_is_open}), $\pi( \tilde{\varphi}_i^{-1}( \pi_i^{-1}(W) ) ) \subseteq U_i$ is in fact open. Also because $\pi$ is open and because it is surjective, there must exist some open subset $V \subseteq \tilde{U}_i$ such that $U_i = \pi(V)$, and so we can simply choose $W$ so that:
                    $$\pi( \tilde{\varphi}_i^{-1}( \pi_i^{-1}(W) ) ) = \varphi_i^{-1}(W)$$
                We have therefore shown that $\varphi_i^{-1}(W) \subseteq U_i$ is open when $W \subseteq \R^n$ is open, i.e. $\varphi_i$ is continuous. Exploiting the fact that $\tilde{\varphi}_i$ is a homeomorphism and that $\pi_i$ is an open map by construction (this is easy to check), we can argue likewise to see that $\varphi_i^{-1}$ is also continuous. Thus, $\varphi_i: U_i \to \R^n$ is a homeomorphism.
            \end{proof}
        \begin{remark}
            While it has been implicitly stated in the proof of proposition \ref{prop: real_projective_spaces_as_topological_manifolds}, let us remark that the charts $(U_i, \varphi_i)$ of $\P^n_{\R}$ are all actually of the form:
                $$(U_i, \varphi_i) = (\tilde{U}_i, \pi_i \circ \tilde{\varphi}_i)$$
            for some chart $(\tilde{U}_i, \tilde{\varphi}_i)$ of $\R^{n + 1} \setminus \{0\}$, as a consequence of $\pi$ being surjective and open.
        \end{remark}
        \begin{corollary}[Projectivisation of abstract finite-dimensional vector spaces] \label{coro: projectivisation_of_finite_dimensional_vector_spaces}
            
        \end{corollary}
            \begin{proof}
                
            \end{proof}
        \begin{proposition}[Real projective spaces are compact] \label{prop: real_projective_spaces_are_compact}
            $\P^n_{\R}$ is compact.
        \end{proposition}
            \begin{proof}
                Simply use the fact that $\R^{n + 1} \setminus \{0\} \cong \bbS^{n + 1}_{\R}$\todo{Add citation.}, then that spheres are compact, and then the fact that quotient maps preserve compactness.
            \end{proof}

    \subsection{Differentiable structures}
        Since manifolds locally resemble Euclidean spaces, it makes sense to speak of derivatives of maps between manifolds, which is a purely local construction.
        \begin{definition}[Differentiable functions] \label{def: differentiable_functions}
            Let $X, Y$ be manifolds and $f: X \to Y$ be a morphism. Such a map is said to be $C^k$ if and only if it admits \textit{continuous} $k^{th}$ order partial derivatives, and if it admits continuous partial derivatives of all orders, then we shall say that it is $C^{\infty}$ or \textbf{smooth}. A $C^k$-function with a $C^k$-inverse is said to be a \textbf{$C^k$-diffeomorphism}. $C^0$ means continuity, and a $C^0$-diffeomorphism is nothing but a homeomorphism.

            The set of $C^k$-functions from $X$ to $Y$ is denoted by $C^k(X, Y)$, and note that when $Y = \R$, this is furthermore a commutative $\R$-algebra via pointwise addition and multiplication.
        \end{definition}
        We remind the reader that one takes the derivative of $f$ at a point $x \in X$ by firstly choosing an open neighbourhood $V \ni f(x)$ along with charts $(V, \psi), (f^{-1}(V), \varphi)$, which identify $V$ and its preimage $f^{-1}(V)$ with, say $\R^m$ and $\R^n$. The rest is vector calculus.

        In light of definition \ref{def: differentiable_functions}, let us consider the following. First, suppose that $X$ is a topological manifold and let $X_{\open}$ denote the small\footnote{This site is small because a manifold has only countably many connected component and is locally compact.} site whose objects are open immersions $U \hookrightarrow X$ and whose morphisms are the commutative triangles in $\Mfd$ between such maps. Next, choose some $k \in \N \cup \{+\infty\}$ consider the following distinguished object of $\Psh(X_{\open})$:
            $$\scrO_X$$
        which is given by:
            $$\scrO_X(U) := C^k(U, \R)$$
        for each $U \in \Ob(X_{\open})$. Note that $\scrO_X$ is indeed a well-defined prsheaf on $X_{\open}$: given open subsets $U, U' \subseteq X$ and a continuous function $\varphi_1: U_1 \to U_2$ compatible with those open immersions, then there will be an $\R$-algebra homomorphism:
            $$\scrO_X(\varphi_1): C^k(U_2, \R) \to C^k(U_1, \R)$$
        (note the contravariance!) given by:
            $$\scrO_X(\varphi_1)(g) := g \circ \varphi$$
        and given two more open subsets $U_2, U_3,  \subseteq X$ and continuous functions $\psi_i: U_i \to U_{i + 1}$, then we will get that:
            $$\scrO_X(\varphi_3 \circ \varphi_2 \circ \varphi_1)(g) := g \circ ( \varphi_3 \circ \varphi_2 \circ \varphi_1 )$$
        from which one easily infers associative. Now, we claim that $\scrO_X$ is actually a sheaf. For convenience, let us for now also say that the pair:
            $$(X, \scrO_X)$$
        as above is a \textbf{differentiable premanifold} and we will be referring to $\scrO_X$ as the \textbf{structure presheaf}.
        \begin{lemma}[Pullbacks of differentiable open immersions] \label{lemma: pullback_of_differentiable_immersions}
            Fix some $k \in \N \cup \{+\infty\}$ and suppose that $X$ is a $C^k$-premanifold. If $\iota_1: U_1 \hookrightarrow U, \iota_2: U_2 \hookrightarrow U$ are open immersions which are also $C^k$, then the canonically induced map $U_1 \cap U_2 \hookrightarrow U$ will also be $C^k$.
        \end{lemma}
            \begin{proof}[Proof sketch]
                Use the fact that open subsets of $n$-manifolds are also $n$-manifolds to identify $U_1, U_2$ with two open $n$-balls.
            \end{proof}
        \begin{corollary}[Differentiable topologies] \label{coro: differentiable_topologies}
            Let $X$ be a $C^k$-premanifold. Then, there shall exist a site $X_{C^k}$ wherein objects are $C^k$-open immersions into $X$, morphisms are commutative triangles of $C^k$-functions between said $C^k$-open immersions, and coverings are given by jointly surjective families of $C^k$-immersions $\{U_i \hookrightarrow U\}_{i \in I}$.

            Furthermore, if $k < l$ then $X_{C^k}$ will be a refinement of $X_{C^l}$, in the sense that if $\scrF$ is a $C^k$-sheaf then it will also be a $C^l$-sheaf. 
        \end{corollary}
            \begin{proof}
                The first assertion comes directly from lemma \ref{lemma: pullback_of_differentiable_immersions}. The second assertion is due to the fact that if $k < l$ then any $C^l$-chart will also be a $C^k$-chart.
            \end{proof}
        We will also usually write $X_{\smooth} := X_{C^{\infty}}$ and refer to this as the \textbf{smooth site} of $X$. Note also that a $C^{\infty}$-manifold can only admit $C^{\infty}$-covers.
        \begin{proposition}[$C^k$-premanifolds are ringed spaces] \label{prop: premanifolds_are_ringed_spaces}
            The structure presheaf of any $C^k$-premanifold (for any $k \in \N \cup \{+\infty\}$) satisfies $C^k$-descent.
        \end{proposition}
            \begin{proof}
                Recall that for $\C$ a site with enough pullbacks and enough representables, and for $F: \C^{\op} \to \calA$ a presheaf with values in some algebraic category $\calA$ (e.g. the categories of groups, rings, modules over a given ring, etc.), such a presheaf is said to satisfy descent if and only if for any object $U \in \Ob(\C)$ and any covering $\{U_a \to U\}_{a \in A}$, the following diagram is an equaliser:
                    $$F(U) \to \prod_{c \in A} F(U_c) \toto \prod_{(a, b) \in A^2} F(U_a \x_U U_b)$$
                In our case where $\C := X_{C^k}$, the pullbacks are intersections of open subsets of $X$ whose open immersions into $X$ are $C^k$ (see proposition \ref{prop: category_of_topological_manifolds}).
            
                Let $U_1, U_2, U_3 \in \Ob(X_{\open})$ be open subsets of $X$ and let $\iota_{a, b}^c: U_a \cap U_b \hookrightarrow U_c$ ($a, b, c \in \{1, 2, 3\}$) be the canonical open immersions (which exists only when $c \in \{a, b\}$). After picking $C^k$-charts $\varphi_{U_a}: U_a \xrightarrow[]{\cong} \R^{n_a}$ (where $a \in \{1, 2, 3\}$), we would like to prove that the immersions $\iota_{a, b}^c$ shall induce $\R$-algebra homomorphisms $C^k(\R^{n_c}, \R) \to C^k(\R^{n_a} \cap \R^{n_b}, \R)$ given by $f \mapsto \varphi_{U_a \cap U_b} \circ f \circ \varphi_{U_c}^{-1}$. Note also that $\varphi_{U_a \cap U_b} = 0$ if and only if $\R^{n_a} \cap \R^{n_b} = \{0\}$ (should we regard $\R^{n_a}, \R^{n_b}$ as subspaces of some $\R^N$ for some $N \geq n_a, n_b$), i.e. if and only if $U_a \cap U_b = \varnothing$. Next, let us choose further identifications $\R^{n_a} \xrightarrow[]{\cong} \B^{n_a}_{\e_a}(x_a)$ for some $x_a \in \R^{n_a}$ and $\e_a > 0$, and let us now also remind the reader that because $U_a \subseteq X$ are open subsets, $n_a = \dim X$ for all $a \in \{1, 2 , 3\}$. The problem then reduces to showing that if $f$ is $C^k$ on two open balls of the same dimension, then it will also be $C^k$ on the intersection. We thus have well-defined $\R$-algebra homomorphisms $C^k(U_c, \R) \to C^k(U_a \cap U_b, \R)$ given by $f \mapsto \varphi_{U_a \cap U_b} \circ f \circ \varphi_{U_c}^{-1}$. We then infer from this that there exists a pair of $\R$-algebra homomorphisms:
                    $$\prod_{c \in A} C^k(U_c, \R) \toto \prod_{(a, b) \in A^2} C^k(U_a \cap U_b, \R)$$
                where $A$ is any arbitrary (countable) indexing set.
                
                To check that $\scrO_X$ is a $C^k$-sheaf, it now remains to check that the equaliser of the diagram above coincides with $C^k(U, \R)$, for every $U \in \Ob(X_{\open})$ with $\{U_a \hookrightarrow U\}_{a \in A}$ as an open cover. This is essentially the statement that if $f: U \to \R$ is $C^k$ on an open subsets $U \subseteq \R^n$ then for every open subset $U' \subseteq U$, the function $f|_{U'}$ will also be $C^k$. This can be verified by first letting the open subsets in questions be open balls of radii $\e > \e'$ repsectively and then using vector calculus.
            \end{proof}

        \begin{definition}[Differentiable manifolds] \label{def: differentiable_manifolds}
            For a given $k \in \N \cup \{+\infty\}$, a \textbf{$C^k$-manifold} is a manifold $X$ equipped with the sheaf of $\R$-algebras $\scrO_X$ on $X_{\open}$ that is given by:
                $$\scrO_X(U) := C^k(U, \R)$$
            A $C^{\infty}$-manifold is also commonly called a \textbf{smooth manifold}.
        \end{definition}

        Like ordinary topological manifolds, there is a category of $C^k$-manifolds for each $k$. To know what the morphisms ought to be, though, let us make the following digression.
        \begin{remark}[Open does not imply smooth!] \label{remark: open_does_not_imply_smooth}
            Let $\iota: W \hookrightarrow X$ be an immersion of a manifold $W$ into another manifold $X$. Should the reader be familiar with schemes, they would have heard of a statement along the lines of:
            
            \begin{center}
                \say{an immersion is smooth (in fact, \'etale) if and only if it is open}
            \end{center}
            \noindent
            In the current setting, however, \textit{there are plenty of open immersions that are not smooth}; in fact, there are many homeomorphisms that are not diffeomorphic (e.g. $\iota: \R \to \R$ given by $\iota(x) := x^n$ for some odd $n$: its continuous inverse is $\iota^{-1}(y) := y^{\frac1n}$, but this is not differentiable at $y = 0$, since $\frac{d}{dy} \iota^{-1}(y) = \frac1n y^{\frac{1 - n}{n}}$ is a formal expression that is undefined at $y = 0$ as $\frac{1 - n}{n} < 0$). We do, though, have that \textit{if $\iota$ is smooth then it will be open}, or more generally, that being $C^k$ implies being open for immersions of manifolds, which leads to proposition \ref{prop: categories_of_differentiable_manifolds}.

            What is slightly confusing is that often, one also finds statements of the kind:
            
            \begin{center}
                \say{any open subset of a smooth manifold has a unique induced smooth structure, and hence is also a smooth manifold}
            \end{center}
            \noindent
            throughout the differential geometry literature, which seemingly contradicts the previous analysis. What is actually happening here is that (and to keep the previous notations) if $X$ is a $C^k$-manifold (for the sake of generality) and if $\iota: W \hookrightarrow X$ is an open immersion, then:
                $$(W, \iota^*\scrO_X)$$
            will be another $C^k$-manifold as along as $\iota$ itself is a $C^k$-function, because we now know through proposition \ref{prop: premanifolds_are_ringed_spaces} that $\scrO_X$ is a sheaf on $X_{\open}$. Because:
                $$\iota^*\scrO_X(U) := \indlim_{ V \supseteq \iota(U) } \scrO_X(V) \cong  \indlim_{ V \supseteq \iota(U) } C^k(V, \R)$$
            which allows us to identify any:
                $$f \in C^k(V, \R)$$
            with:
                $$f \circ \iota \in \iota^*\scrO_X(U)$$
            we thus see clearly that $f \circ \iota$ can not be $C^k$, and therefore can not be a well-defined element of $\iota^*\scrO_X(U)$, when $\iota$ is not $C^k$.

            In conclusion, morphisms of $C^k$-manifolds ought to be $C^k$-functions.

            We would also like to point out one other crucial reason for morphisms of $C^k$-manifolds to be $C^k$-functions: there should be $C^k$-manifolds whose underlying topological spaces are homeomorphic, but nevertheless are non-isomorphic as $C^k$-manifolds. The function $\iota: \R \to \R$ given by $\iota(x) := x^n$ for $n$ odd, as above, is again an example. 
        \end{remark}
        \begin{proposition}[Categories of differentiable manifolds] \label{prop: categories_of_differentiable_manifolds}
            For each $k \in \N \cup \{+\infty\}$, there exists a category $C^k\-\Mfd$ with $C^k$-manifolds as objects and $C^k$-maps between them as morphisms.
        \end{proposition}
            \begin{proof}
                Clear from the fact that compositions of $C^k$-functions are once more $C^k$.
            \end{proof}
        Note also that $C^0\-\Mfd = \Mfd$, tautologically. As for the limits and colimits that can exist in the categories $C^k\-\Mfd$, we will touch on the topic shortly.
        
        Actually, $C^k$-manifolds are not merely ringed spaces, but \textit{locally} ringed spaces. Before phrasing this properly as a proposition, let us note beforehand that if $(X, \scrO_X)$ is a $C^k$-manifold and $x \in X$ is any point, and $U \ni x$ is an open neighbourhood thereof, then the evaluation map:
            $$\ev_x: C^k(U, \R) \to \R$$
        given by:
            $$\ev_x(f) := f(x)$$
        is in fact a surjective $\R$-algebra homomorphism. this is because for every $c \in \R$, there exists the constant function $\underline{c}: U \to \R$ which is trivially $C^k$, and hence a well-defined element of $C^k(U, \R)$. 
        \begin{proposition}[$C^k$ manifolds are locally ringed spaces] \label{prop: differentiable_manifolds_are_locally_ringed_spaces}
            Fix some $k \in \N \cup \{+\infty\}$.
            \begin{enumerate}
                \item Let $(X, \scrO_X)$ be a $C^k$-manifold. Then, for every $x \in X$, the stalk:
                    $$\scrO_{X, x} := \indlim_{U \in \Ob(X_{\open}), U \ni x} \scrO_X(U)$$
                of the structure sheaf will be a local $\R$-algebra: namely, the unique maximal ideal $\m_{X, x} \subset \scrO_{X, x}$ is spanned by (smooth) functions $f \in C^k(U, \R)$ (with $U \ni x$ being some open neighbourhood) such that $f(x) = 0$, i.e. it is the kernel of the evaluation map $\ev_x: C^k(U, \R) \to \R$ given by $\ev_x(f) := f(x)$. 
                \item Furthermore, $C^k\-\Mfd$ is in fact a subcategory of $\LRS$, i.e. every morphism of $C^k$ manifolds is a morphism of locally ringed spaces\footnote{Though, the converse isn't necessarily true. See remark \ref{remark: open_does_not_imply_smooth}.}.
            \end{enumerate}
        \end{proposition}
            \begin{proof}
                \begin{enumerate}
                    \item Clear from the surjectivity of the $\R$-algebra homomorphism $\ev_x$.
                    \item The only thing to show is that if $\varphi: X \to Y$ is a morphism in $C^k\-\Mfd$, and $x \in X$ is any point, then the $\R$-algebra homomorphism $\varphi^{\sharp}_x: \scrO_{Y, \varphi(x)} \to \scrO_{X, x}$ - canonically induced by the pushforward map $\varphi^{\sharp}: \scrO_Y \to \varphi_* \scrO_X$ - is local; for the sake of thoroughness, let us note that due to the functoriality of colimits, $\varphi^{\sharp}_x$ is still given by $g \mapsto g \circ \varphi$ for every $g \in C^k(U, \R)$ where $U \ni x$ is an open neighbourhood. If $g \in \m_{Y, \varphi(x)}$ then $g(\varphi(x)) = 0$ per the previous assertion, but this equivalently means that $\varphi^{\sharp}_x(g)(x) = 0$. As such, $\varphi^{\sharp}_x( \m_{Y, \varphi(x)}) \subseteq \m_{X, x}$, and hence $\varphi^{\sharp}_x$ is indeed a local $\R$-algebra homomorphism.
                \end{enumerate}
            \end{proof}

        \begin{convention}[(Quasi-)affine spaces] \label{conv: (quasi)_affine_spaces}
            Let $n \in \N$. When the Euclidean-normed space $\R^n$ is being considered as a smooth real manifold of dimension $n$, we will sometimes refer to it as the \textbf{real affine $n$-space}. We avoid the term \textbf{affine manifold} since it has another established meaning.

            A \textbf{quasi-affine $n$-space} shall be an open subspace of $\R^n$, endowed with the restricted smooth structure.
        \end{convention}

        \begin{example}[Punctured affine spaces as smooth manifolds] \label{example: punctured_affine_spaces_as_smooth_manifolds}
            Consider the punctured real affine $n$-space $\R^{n + 1} \setminus \{0\}$ with the subspace topology, as in example \ref{example: punctured_affine_spaces_as_topological_manifolds}. The canonical inclusion:
                $$\iota: \R^{n + 1} \setminus \{0\} \to \R^{n + 1}$$
            of this open subspace into $\R^{n + 1}$ is smooth, since the domain restriction of any smooth function remains smooth, so according to remark \ref{remark: open_does_not_imply_smooth}, $\R^{n + 1} \setminus \{0\}$ will have the structure of a smooth submanifold of $\R^{n + 1}$ via restriction of the structure sheaf along $\iota$:
                $$\scrO_{\R^{n + 1} \setminus \{0\}} := \iota^* \scrO_{\R^{n + 1}}$$
            This makes $\R^{n + 1} \setminus \{0\}$ into a quasi-affine $n$-space.
        \end{example}

    \subsection{Partitions of unity and soft sheaves}
        \begin{definition}[Partitions of unity] \label{def: partitions_of_unity}
            A \textbf{$C^k$-partition of unity} of a $C^k$-manifold $X$ is the data of a $C^k$-cover $\{U_i \hookrightarrow X\}_{i \in I}$ (in the sense of corollary \ref{coro: differentiable_topologies}) along with a family of $C^k$-functions $\{f_i: U_i \to [0, 1]\}_{i \in I}$ such that:
            \begin{itemize}
                \item for all points $x \in X$:
                    $$\sum_{i \in I} f_i(x) = 1$$
                \item each $\supp f_i := \overline{ \{ x \in U_i \mid f_i(x) \not = 0 \} }$ is a para-compact subset of $U_i$.
            \end{itemize}
        \end{definition}
        \begin{lemma}[Para-compact spaces have partitions of unity] \label{lemma: para_compact_spaces_have_partitions_of_unity}
            A Hausdorff topological space is para-compact if and only if it admits a $C^0$-partition of unity. 
        \end{lemma}
            \begin{proof}
                Let $X$ be a Hausdorff topological space.
            
                Suppose firstly that $X$ is para-compact. Because $X$ is Hausdorff, it admits a disjoint partition:
                    $$X := \coprod_{i \in I} U_i$$
                into open subsets $U_i$. The sought-for $C^0$-partition of unity can then be taken to be the family of characteristic functions $\chi_{U_i}: U_i \to [0, 1]$ which are given by:
                    $$
                        \chi_{U_i}(x) :=
                        \begin{cases}
                            \text{$1$ if $x \in U_i$}
                            \\
                            \text{$0$ if $x \not \in U_i$}
                        \end{cases}
                    $$
                and hence are continuous and such that $\supp \chi_{U_i} \subset U$ by construction. As $X$ is para-compact, so are the subsets $\supp \chi_{U_i}$ (with respect to their subspace topologies). Since each point $x \in X$ can only belong to one of the open subsets $U_i$, we also have that:
                    $$\sum_{i \in I} \chi_{U_i}(x) = 1$$
                for all $x \in X$. As such, we have constructed a $C^0$-partition of unity for $X$.

                Conversely, suppose that $X$ admits a $C^0$-partition of unity $\{f_i: U_i \to [0, 1]\}_{i \in I}$. Per the definition of partitions of unity, $\{U_i \to X\}_{i \in I}$ is an open cover of $X$, and because $X$ is Hausdorff, we can assume that the open subsets $U_i$ are disjoint, i.e.:
                    $$X := \coprod_{i \in I} U_i$$
                which, in particular, implies that:
                    $$\forall i, j \in I: \supp f_i \cap \supp f_j = \varnothing$$
                since $\supp f_i \subseteq U_i$. Now, for any continuous function $f: U \to \R$, note that:
                    $$(X \setminus \supp f) \cap U = U \setminus \supp f := \{ x \in U \mid f(x) = 0 \}$$
                from which we gather that:
                    $$X \setminus \coprod_{i \in I} \supp f_i = (X \setminus \coprod_{i \in I} \supp f_i) \cap \coprod_{i \in I} U_i = \coprod_{i \in I} (U_i \setminus \supp f_i)$$
                Now, let $\mu_i: \calB(U_i) \to \R_{\geq 0}$ be the Borel measure. Then:
                    $$\int_{U_i} f d\mu_i = \int_{\supp f_i} f d\mu_i$$
                for all measureable functions $f: U_i \to \R$. The above shows that:
                    $$\mu_i(U_i \setminus \supp f_i) = \int_{U_i} \chi_{U_i \setminus \supp f_i} d\mu_i = 0$$
                This shows that the interior of $U_i \setminus \supp f_i$ is empty, but $U_i \setminus \supp f_i$ is an open subset of $U_i$ (since $\supp f_i$ is closed by definition), and hence:
                    $$U_i \setminus \supp f_i = \varnothing$$
                Per the definition of partitions of unity, $\supp f_i$ is para-compact, and since the open subsets $U_i$ together form an open cover of $X$, we have managed to show that $X$ is para-compact.
            \end{proof}
        Since topological manifolds are para-compact \textit{a priori}, lemma \ref{lemma: para_compact_spaces_have_partitions_of_unity} implies in particular that topological manifolds admit $C^0$-partitions of unity. In fact, this holds true for all $C^k$-manifolds, for all $k \in \N \cup \{+\infty\}$.
        \begin{proposition}[Existence of partitions of unity] \label{prop: existence_of_partitions_of_unity}
            For any $k \in \N \cup \{+\infty\}$, any $C^k$-manifold admits a partition of unity.
        \end{proposition}
            \begin{proof}
                Characteristic functions are smooth, and $C^k$-manifolds are para-compact \textit{a priori}, and the assertion follows.
            \end{proof}
    
        \begin{theorem}[Existence of partitions of unity implies acyclicity of structure sheaves]
            Let $X$ be a $C^k$-manifold, for some $k \in \N \cup \{+\infty\}$. Then, the structure sheaf $\scrO_X$ is acyclic.
        \end{theorem}
            \begin{proof}
                
            \end{proof}
        \begin{corollary}[Domain expansions of smooth functions]
            
        \end{corollary}

        \begin{definition}[Coherent modules] \label{def: coherent_modules}
            
        \end{definition}

    \subsection{Tangent spaces}
        \begin{definition}[Tangent spaces] \label{def: tangent_spaces}
            Let $X$ be a $C^k$-manifold (for some $k \in \N \cup \{+\infty\}$). Then, the \textbf{tangent space} at a point $x \in X$ shall be the vector space of degree-$1$ differential operators on $\scrO_{X, x}$ with values in $\R$, i.e.:
                $$T_{X, x} := \Diff^1_{\R}( \scrO_{X, x}, \R )$$
            Elements of tangent spaces are usually called \textbf{vector fields}.
        \end{definition}
        By picking local coordinates $x := (x_1, ..., x_{\dim X})$ for an open neighbourhood $U_x \ni x$ which is sufficiently small so that $\scrO_{X, x} \cong C^k(U_x)$, we can also identify:
            $$T_{X, x} \cong \bigoplus_{i = 1}^{\dim X} C^k(U_x)/\m_{X, x} \cdot \frac{\del}{\del x_i} |_x \cong \bigoplus_{i = 1}^{\dim X} \scrO_{X, x}/\m_{X, x} \cdot \frac{\del}{\del x_i} |_x \cong \bigoplus_{i = 1}^{\dim X} \R \cdot \frac{\del}{\del x_i} |_x$$
        wherein $(-) |_x$ means evaluation at $x$. Note that the coefficients are taken modulo $\m_{X, x}$ because this (maximal) ideal of $\scrO_{X, x}$ is spanned by $C^k$-functions $f: U_x \to \R$ that vanish at $x$, and hence summands with such functions as coefficients will not matter anyway. From the above, it is trivial to see that at all points $x \in X$ of a $C^k$-manifold, we have that:
            $$\dim T_{X, x} = \dim X$$

        Alternatively, we can characterise the tangent space at $x \in X$ by:
            $$T_{X, x} \cong \Der_{\R}(\scrO_{X, x}, \R)$$

        For what follows, recall that for any homomorphism of commutative rings $R \to S$, the $S$-module of differential $1$-forms is given by:
            $$\Omega^1_{S/R} := (S \tensor_R S)/J_{S/R}$$
        with $J_{S/R} \subset S \tensor_R S$ is the ideal generated by elements of the form:
            $$fg \tensor 1 - f \tensor g - g \tensor f$$
        given for all $f, g \in S$. Such modules are the objects that corepresent the functor:
            $$\Der_R(S, -): S\mod \to S\mod$$
        in the sense that there is a natural isomorphism:
            $$\Hom_S(\Omega^1_{S/R}, -) \cong \Der_R(S, -)$$
        For more details, see \cite[\href{https://stacks.math.columbia.edu/tag/00RM}{Tag 00RM}]{stacks}.
        \begin{lemma}[Cotangent sheaves and cotangent bundles] \label{lemma: cotangent_sheaves_and_budnles}
            For every $C^k$-manifold $X$, there is a coherent\footnote{In fact, a vector bundle, but we will come back to this later.} $\scrO_X$-module $\Omega^1_X$ given for every $C^k$-open immersion $U \subseteq X$ by:
                $$\Omega^1_X(U) := \Omega^1_{C^k(U)/\R}$$
        \end{lemma}
            \begin{proof}
                
            \end{proof}
        \begin{corollary}[Tangent sheaves and tangent bundles] \label{coro: tangent_sheaves_and_bundles}
            
        \end{corollary}
        \begin{lemma}[Cotangent spaces]
            
        \end{lemma}
            \begin{proof}
                
            \end{proof}
        \begin{convention}
            To conform to a traditional convention in the theory of integral equations, in calculus of variations, and in probability theory, the arguments of functions whose inputs are linear functionals will be placed between square brackets $[]$ as opposed to the usual parentheses $()$. For example, if $V, W$ are vector spaces and $T: V \to W$ is a linear map then the evaluation of the dual map $T^*: W^* \to V^*$ on functionals $\psi \in W^*$ will be written:
                $$T[\psi^*]$$
            instead of $T(\psi^*)$. This may also help with readability in certain cases. 
        \end{convention}
        \begin{corollary}[Functoriality of tangent spaces] \label{coro functoriality_of_tangent_spaces}
            Any morphism $\varphi: X \to Y$ of $C^k$-manifolds covariantly induces a linear map between tangent spaces:
                $$d\varphi_x: T_{X, x} \to T_{Y, \varphi(x)}$$
            that is determined for all $v \in \Der_{\R}(\scrO_{X, x}, \R)$ and all $g \in \scrO_{Y, \varphi(x)}$ by:
                $$d\varphi_x( v )[g] := v|_{\varphi(x)}[g \circ \varphi]$$
            Note that if $\varphi$ is not differentiable, then the RHS may not be well-defined, as $v$ acts via partial derivatives, which is why we need to assume this. 
        \end{corollary}
            \begin{proof}
                
            \end{proof}
            
        \begin{definition}[Jacobians] \label{def: jacobian}
            For any morphism $\varphi: X \to Y$ of $C^k$-manifolds, the $\dim Y \x \dim X$ matrix that represents $d\varphi_x: T_{X, x} \to T_{Y, \varphi(x)}$ - with respect to some basis choices for $T_{X, x}$ and $T_{Y, \varphi(x)}$ - is called the \textbf{Jacobian} at $x$ of $\varphi$. In terms of local coordinates $x := (x_1, ..., x_{\dim X})$ and $\varphi(x) := (y_1, ..., y_{\dim Y})$, we have that:
                $$
                    d\varphi_x =
                    \begin{pmatrix}
                        \nabla \varphi_1
                        \\
                        \vdots
                        \\
                        \nabla \varphi_{\dim Y}
                    \end{pmatrix}
                $$
            wherein $\varphi := (\varphi_1, ..., \varphi_{\dim Y})$ is the component representation, and each row of $d\varphi_x$ is given by $\nabla \varphi_j := \left( \frac{\del \varphi_j}{\del x_i} \right)_{1 \leq i \leq \dim X}$ for each $1 \leq j \leq \dim Y$.
        \end{definition}

        \begin{lemma}[\texorpdfstring{Tangent spaces as spaces of $\R[\e]/\e^2$-points}{}] \label{lemma: tangent_spaces_and_dual_numbers}
            Let $k \geq 1$ and let $X$ be a $C^k$-manifold. Then, there shall be an isomorphism of $\R$-vector spaces:
                $$T_{X, x} \cong \Mor_{\LRS_{/\A^{1, \smooth}_{\R}}}(\Spec \R[\e]/\e^2, X)$$
        \end{lemma}
            \begin{proof}
                Since $\R[\e]/\e^2$ is a local $\R$-algebra, it is enough to prove that there is a linear isomorphism:
                    $$\Phi: T_{X, x} \xrightarrow[]{\cong} \Hom_{\R\-\Comm\Alg}(\scrO_{X, x}, \R[\e]/\e^2)$$

                Firstly, note that $\Hom_{\R\-\Comm\Alg}(\scrO_{X, x}, \R[\e]/\e^2)$ is indeed a vector space via pointwise addition and scalar multiplication.
                    
                Since $\R[\e]/\e^2 \cong \R \cdot 1 \oplus \R \cdot \e$ as a vector space, any $\R$-algebra homomorphism $\varphi: \scrO_{X, x} \to \R[\e]/\e^2$ must be given by:
                    $$\varphi := \varphi_0 \oplus \varphi_1$$
                for some linear functionals $\varphi_0, \varphi_1 \in \scrO_{X, x}^*$. If we let $\ev_x: \scrO_{X, x} \to \R$ be the evaluation at $x$, i.e.:
                    $$\ev_x[f] := f|_x$$
                then we claim that $\Phi$ can be given by:
                    $$\Phi(\xi) := (\ev_x \circ \xi^0) \oplus (\ev_x \circ \xi)$$
                should we regard $\xi$ as a derivation on $\scrO_{X, x}$. $\Phi$ is therefore automatically linear, so let us focus on showing that it is bijective. 
                
                If we let $\m_{X, x} := \ker \ev_x$, then $\xi \cdot f \in \m_{X, x}$ if and only if $f \in \m_{X, x}^2$, and because $\m_{X, x}^2 \subset \m_{X, x}$, we see that $\ker \Phi = (\m_{X, x}^2)^*$. However, because $T_{X, x} \cong ( \m_{X, x}/ \m_{X, x}^2 )^* \cong \m_{X, x}^*/(\m_{X, x}^2)^*$ with the latter isomorphism holding by finite-dimensionality, we actually have that $\ker \Phi = 0$, i.e. $\Phi$ is injective. Since $\xi$ acts as a derivation on $\scrO_{X, x}$, the map $\Phi$ is surjective if and only if every $f \in \scrO_{X, x}$ is partially integrable, but this is always the case for $C^k$-functions, so $\Phi$ is indeed surjective. We have thus shown that $\Phi$ is both injective and surjective, and hence bijective. 
            \end{proof}
        \begin{corollary}[Tangent spaces of products] \label{coro: tangent_space_of_products}
            Let $k \geq 1$ and let $X, Y$ be $C^k$-manifolds. Then, at any point $(x, y) \in X \x Y$, we have an isomorphism of $\R$-vector spaces:
                $$T_{X \x Y, (x, y)} \cong T_{X, x} \x T_{Y, y}$$
        \end{corollary}
            \begin{proof}
                Clear from lemma \ref{lemma: tangent_spaces_and_dual_numbers} and the fact that hom-functors preserve finite limits, which gives:
                    $$
                        \begin{aligned}
                            T_{X \x Y, (x, y)} & \cong \Mor_{\LRS_{/\A^{1, \smooth}_{\R}}}(\Spec \R[\e]/\e^2, X \x Y)
                            \\
                            & \cong \Mor_{\LRS_{/\A^{1, \smooth}_{\R}}}(\Spec \R[\e]/\e^2, X) \x \Mor_{\LRS_{/\A^{1, \smooth}_{\R}}}(\Spec \R[\e]/\e^2, Y)
                            \\
                            & \cong T_{X, x} \x T_{Y, y}
                        \end{aligned}
                    $$
            \end{proof}

    \subsection{Local properties of morphisms and some (co)limits of differentiable manifolds}
        \begin{lemma}[Locally constant morphisms] \label{lemma: locally_constant_morphisms}
            A morphism of $C^k$-manifolds $\varphi: X \to Y$ is locally constant if and only if for all $x \in X$, the differential $df_x: T_{X, x} \to T_{Y, \varphi(x)}$ is the zero map.
        \end{lemma}
            \begin{proof}[Coordinate-dependent proof]
                For any $x \in X$, by picking bases:
                    $$\left\{ \frac{\del}{\del x_i} \right\}_{1 \leq i \leq \dim X} \subset T_{X, x}, \left\{ \frac{\del}{\del y_j} \right\}_{1 \leq j \leq \dim Y} \subset T_{Y, \varphi(x)}$$
                one sees via explicitly writing down the matrix representing $d\varphi_x$, which is nothing but the Jacobian at $x$ of $\varphi$, that this matrix is $0$ if and only if $\varphi$ is constant on a $C^k$-open neighbourhood of $x$.
            \end{proof}
            \begin{proof}[Coordinate-free proof]
                To begin, recall that by definition, $d\varphi_x: T_{X, x} \to T_{Y, \varphi(x)}$ is given by:
                    $$d\varphi_x(v)[g] := v[g \circ \varphi]$$
                for all $v \in T_{Y, \varphi(x)}$ and all $g \in \scrO_{Y, \varphi(x)}$. Assume firstly that $\varphi: X \to Y$ is constant. This forces $g \circ \varphi$ to also be constant, and because $v$ acts as a derivation on $\scrO_{Y, y}$, this forces $v[g \circ \varphi] = 0$ at all points $x \in X$. Conversely, suppose that $d\varphi_x = 0$ for all $x \in X$. This means that $0 = d\varphi_x(v)[g] := v[g \circ \varphi]$ for all $v \in T_{Y, \varphi(x)}$ and all $g \in \scrO_{Y, y}$. This implies that $g \circ \varphi$ must be constant for all $g \in \scrO_{Y, y}$, and therefore $\varphi$ must be constant on a $C^k$-open neighbourhood of $x$.  
            \end{proof}
    
        \begin{definition}[Submersive, unramified, and \'etale morphisms] \label{def: submersive_unramified_and_etale_morphisms}
            A morphism of $C^k$-manifolds, say $\varphi: X \to Y$ is said to be:
            \begin{itemize}
                \item \textbf{submersive} at $x \in X$ if and only if $d\varphi_x$ is surjective,
                \item \textbf{unramified} at $x \in X$ if and only if $d\varphi_x$ is injective (cf. \cite[\href{https://stacks.math.columbia.edu/tag/0B2G}{Tag 0B2G}]{stacks}), and
                \item \textbf{\'etale} at $x \in X$ if and only if it is both submersive and unramified, i.e. $d\varphi_x$ is bijective.
            \end{itemize}

            A morphism $\varphi: X \to Y$ is referred to plainly as being \textbf{submersive/unramified/\'etale} if and only if it is so at every point $x \in X$.
        \end{definition}
        \begin{convention}[Immersions]
            Often, in the differential geometry literature, the domain of an \textbf{immersion} is not assumed to be homeomorphic to its image. However, in algebraic geometry, one always assume that immersions are homeomorphisms onto their images\footnote{At least, \cite{stacks} does ...}. Thus, to avoid linguistic conflicts, let us say that a morphism of topological manifolds $f: X \to Y$ is an \textbf{immersion}\footnote{\cite{lee_smooth_manifolds} calls such morphisms \say{embeddings}.} if and only if the underlying set map is injective.
            
            An \textbf{embedding} in the sense of \cite{lee_smooth_manifolds} is an \textit{unramified} immersion between differentiable manifolds. It is important to keep in mind the unramified morphisms, when regarded as maps between the underlying topological spaces, are not injective in general.
        \end{convention}
        \begin{convention}
            An \'etale morphism between two $C^k$-manifolds is also called a \textbf{local $C^k$-diffeomorphism}.
        \end{convention}

        \begin{lemma}[Compositions of submersive/unramified/\'etale morphisms] \label{lemma: compositions_of_submersive_unramified_and_etale_morphisms}
            Suppose that:
                $$\varphi: X \to Y, \psi: Y \to Z$$
            are morphisms of $C^k$-manifolds. If $\varphi$ is submersive/unramified/\'etale at a point $x \in X$ and $\psi$ is respectively submersive/unramified/\'etale at the point $\varphi(x) \in Y$, then $\psi \circ \varphi$ will also be submersive/unramified/\'etale at $x \in X$.
        \end{lemma}
            \begin{proof}
                Clear from definition \ref{def: submersive_unramified_and_etale_morphisms} and the functoriality of tangent spaces established in corollary \ref{coro functoriality_of_tangent_spaces}.
            \end{proof}

        \begin{theorem}[Rank theorem] \label{theorem: rank_theorem} 
            Suppose that $\varphi: X \to Y$ is a morphism of $C^k$-manifolds and suppose that $\dim(\im d\varphi_x) = r$. Then, there shall exist $C^k$-charts $x \in U \cong \R^{\dim X}$ and $\varphi(x) \in V \cong \R^{\dim Y}$ with local coordinates $x := (x_1, ..., x_{\dim X})$ and $\varphi(x) := (y_1, ..., y_{\dim Y})$ such that:
                $$\varphi|_U: \R^{\dim X} \to \R^{\dim Y}$$
            is given by:
                $$
                    \varphi|_U(x_1, ..., x_{\dim X}) =
                    \begin{cases}
                        \text{$(y_1, ..., y_{r = \dim Y})$ if $\varphi$ is submersive at $x \in X$}
                        \\
                        \text{$(y_1, ..., y_r, 0, ..., 0)$ if $\varphi$ is unramified at $x \in X$}
                    \end{cases}
                $$
        \end{theorem}
            \begin{proof}
                Charts are globally \'etale by definition (cf. definitions \ref{def: atlases_and_charts} and \ref{def: submersive_unramified_and_etale_morphisms}), and observe that the composition of a submersive/unramified morphism with a globally \'etale one remains submersive/unramified respectively. Next, $\varphi|_U$ when given in terms of local coordinates is nothing but the composition $\R^{\dim X} \xrightarrow[]{\cong} U \xrightarrow[]{\varphi|_U} V \xrightarrow[]{\cong} \R^{\dim Y}$, with the outer two arrows being globally \'etale by virtue of being charts. Furthermore, $T_{U, x} \cong T_{\R^{\dim X}, (x_1, ..., x_{\dim X})} \cong \R^{\dim X}$ and $T_{V, \varphi(x)} \cong T_{\R^{\dim Y}, (y_1, ..., y_{\dim Y})}$. Thus, if $\varphi$ is submersive/unramified at $x \in X$, then the Jacobian:
                    $$d\varphi_x: T_{\R^{\dim X}, (x_1, ..., x_{\dim X})} \to T_{\R^{\dim Y}, (y_1, ..., y_{\dim Y})}$$
                will be of non-full\footnote{We do not claim that not being of full rank implies being injective!}/full rank respectively. We can choose appropriate bases for the tangent spaces such that $d\varphi_x$ is in RREF, which then gives the claim, since a matrix in RREF has a zero row if and only if it is not of full rank.
            \end{proof}
        \begin{corollary} \label{coro: submersive_and_unramified_morphisms_are_open}
            Submersions and unramified immersions are open.
        \end{corollary}
            \begin{proof}
                
            \end{proof}
        \begin{theorem}[Inverse Function Theorem] \label{theorem: inverse_function_theorem}
            Suppose that $\varphi: X \to Y$ is a morphism of $C^k$-manifolds. If $d\varphi_x: T_{X, x} \to T_{Y, \varphi(x)}$ is a linear isomorphism then there will exist \underline{connected} $C^k$-open neighbourhoods $U \ni x$ and $V \ni \varphi(x)$ such that $\varphi|_U: U \to V$ is \'etale.
        \end{theorem}
            \begin{proof}
                
            \end{proof}

        \begin{lemma}[Regular epimorphisms of differentiable manifolds] \label{lemma: regular_epimorphisms_of_differentiable_manifolds}
            Surjective $C^k$-submersions are regular epimorphisms in $C^k\-\Mfd$. Consequently, every $C^k$-submersion admits local $C^k$-sections.
        \end{lemma}
            \begin{proof}
                
            \end{proof}

        \begin{example}[Real projective spaces as smooth manifolds] \label{example: real_projeective_spaces_as_smooth_manifolds}
            We claim that the topological $n$-manifold $\P^n_{\R}$ as in example \ref{example: real_projective_spaces_as_topological_manifolds} has a canonical smooth structure induced by the quotient map $\pi: \R^{n + 1} \setminus \{0\} \to \P^n_{\R}$. Moreover, we claim that $\pi$ is a smooth submersion.
            
            Consider two standard charts $(U_i, \varphi_i), (U_j, \varphi_j)$ for the topological $n$-manifold $\P^n_{\R}$ as in the proof of proposition \ref{prop: real_projective_spaces_as_topological_manifolds}.
            \todo[inline]{Show that $\pi$ is smooth.}

            Now, to show that $\pi$ is submersive, consider its differential at a point $w \in U_i$ where $(U_i, \varphi_i)$ is a standard (smooth) chart of $\P^n_{\R}$, with local coordinates $\varphi_i([w_1 : ... : w_n]) := \left( \frac{w_1}{w_i}, ..., \frac{w_{i - 1}}{w_i}, \frac{w_{i + 1}}{w_i}, ..., \frac{w_n}{w_i} \right)$:
                $$d\pi_w = \left( \nabla\left( \frac{w_j}{w_i} \right) \right)_{1 \leq j \not = i \leq n}$$
            By direct computation (during which one will be making use of the fact that $(w_1, ..., w_n) \not = (0, ..., 0)$), one easily sees that this matrix is of full rank, hence $d\pi_w$ is surjective everywhere, i.e. $\pi$ is a submersion.
        \end{example}
        \begin{convention}[Quasi-projective spaces] \label{conv: quasi_projective_spaces}
            Open subspaces of $\P^n_{\R}$, when endowed with the smooth structures obtained via restricting $\scrO_{\P^n_{\R}}$, shall be referred to as \textbf{quasi-projective $n$-spaces}. 
        \end{convention}

        \begin{definition}[Proper maps] \label{def: proper_maps}
            A continuous map $\pi: X \to S$ between topological spaces is said to be \textbf{proper} if and only if $\pi^{-1}(K) \subseteq X$ is compact for every compact subspace $K \subseteq S$.
        \end{definition}
        \begin{lemma}[Compact if and only if proper over a point] \label{lemma: compact_iff_proper_over_a_point}
            Let $X$ be a topological space, let $\pt$ be the one-point space, and let $\pi: X \to \pt$ be the terminal map. Then, $X$ will be compact if and only if $\pi$ is proper.
        \end{lemma}
            \begin{proof}
                If $\pi$ is proper then because $\pi^{-1}(\pt) = X$ and because $\pt$ is compact, $X$ will automatically be compact. Conversely, if $X$ is compact, then again because the preimages of the only two subsets of $\pt$, namely $\pt$ itself and $\varnothing$ (both of which are compact) are $X$ and $\varnothing$ respectively, $\pi$ will be proper by definition.
            \end{proof}
        \begin{lemma}[Closed Map Lemma] \label{lemma: closed_map_lemma}
            Let $\pi: X \to S$ be a continuous map from a compact space to a Hausdorff space. Then, $\pi$ will be closed and proper.
        \end{lemma}
            \begin{proof}
                \todo[inline]{Closed Map Lemma}
            \end{proof}
        There is a vague dichotomy between affine and projective spaces, in the following sense.
        \begin{theorem}[Compact differentiable manifolds do not submerge] \label{theorem: compact_differentiable_manifolds_do_not_submerge}
            If $X$ is a non-empty and compact $C^k$-manifold, then there will be no $C^k$-submersion $\pi: X \to \R^n$ for any $n \geq 1$. Consequently, one can never find any submersion from a \underline{compact} quasi-projective space into any quasi-affine space of dimension $\geq 1$.
        \end{theorem}
            \begin{proof}
                Suppose for the sake of deriving a contradiction that there exists a submersion $\pi: X \to \R^n$ for some $n \geq 1$. By lemma \ref{lemma: regular_epimorphisms_of_differentiable_manifolds}, we may as well assume that $\pi$ is a regular epimorphism in $C^k\-\Mfd$, which in particular makes it surjective, i.e. $X \cong \pi^{-1}(\R^n)$. Now, by lemma \ref{lemma: closed_map_lemma}, we know that $\pi$ must be proper, since $X$ is compact by assumption and $\R^n$ is Hausdorff by virtue of being a metric space. Since $\R^n$ is not compact, the contraposition of lemma \ref{lemma: compact_iff_proper_over_a_point} then stipulates that $\pi^{-1}(\R^n)$ must also be non-compact. Since $X$ is compact and $X \cong \pi^{-1}(\R^n)$ by our assumption, we thus have a contradiction, and hence $\pi$ can not be submersive. 
            \end{proof}
        \begin{example}
            There are neither submersions $\bbS^n \to \R^{n + 1} \setminus \{0\}$ nor $\bbD^n \to \R^n$. One can not submerge any compact Lie group into any $\GL_n(\R)$ either.
        \end{example}

        Let us now turn our attention towards certain limits of $C^k$-manifolds, namely pullbacks. These limits are of particular importance, because they are intersections that exist in the category of $C^k$-manifolds.
        \begin{remark}[A reminder about pullbacks in abelian categories] \label{remark: pullbacks_in_abelian_categories}
            For what follows, recall that in an abelian category $\calA$, the kernel of a morphism $f: M \to N$ can be formed as the following pullback:
                $$
                    \begin{tikzcd}
                	{\ker f} & M \\
                	0 & N
                	\arrow[from=1-1, to=1-2]
                	\arrow[from=1-1, to=2-1]
                	\arrow["\lrcorner"{anchor=center, pos=0.125}, draw=none, from=1-1, to=2-2]
                	\arrow["f", from=1-2, to=2-2]
                	\arrow[from=2-1, to=2-2]
                    \end{tikzcd}
                $$
            In general, limits are formed by composing products and equalisers, and since the latter are kernels in abelian categories, pullbacks in abelian categories must arise as kernels of morphisms whose domains are products. 
            
            Recall now that given any pair of objects $M_1, M_2 \in \Ob(\calA)$, to give a pair of morphisms $f_1: M_1 \to N, f_2: M_2 \to N$ is the same as to give a morphism $f_1 \x f_2: M_1 \x M_2 \to N$, which fits into the following commutative diagram:
                $$
                    \begin{tikzcd}
                	{M_1 \x M_2} & {M_2} \\
                	{M_1} & N
                	\arrow["{\pr_2}", from=1-1, to=1-2]
                	\arrow["{\pr_1}"', from=1-1, to=2-1]
                	\arrow["{f_1 \x f_2}"{description}, dashed, from=1-1, to=2-2]
                	\arrow["{f_2}", from=1-2, to=2-2]
                	\arrow["{f_1}"', from=2-1, to=2-2]
                    \end{tikzcd}
                $$
            We now claim that:
                $$M_1 \x_{f_1, N, f_2} M_2 \cong \ker(f_1 \x f_2)$$
            By definition, $M_1 \x_{f_1, N, f_2} M_2$ is terminal amongst all objects $P \in \Ob(\calA)$ with morphisms $\pi_1: P \to M_1, \pi_2: P \to M_2$ such that $f_i \circ \pi_i$, while $\ker(f_1 \x f_2)$ is terminal amongst all objects $K \in \Ob(\calA)$ with a morphism $\kappa: K \to M_1 \x M_2$ such that $0 = (f_1 \x f_2) \circ \kappa = (f_i \circ \pr_i) \circ \kappa$. These two universal properties forces there two be a unique isomorphism $M_1 \x_{f_1, N, f_2} M_2 \cong \ker(f_1 \x f_2)$, thus proving the claim.

            Consequently, we have the following pullback:
                $$
                    \begin{tikzcd}
                	{M_1 \x_{f_1, N, f_2} M_2} & {M_1 \x M_2} \\
                	0 & N
                	\arrow[from=1-1, to=1-2]
                	\arrow[from=1-1, to=2-1]
                	\arrow["\lrcorner"{anchor=center, pos=0.125}, draw=none, from=1-1, to=2-2]
                	\arrow["{f_1 \x f_2}", from=1-2, to=2-2]
                	\arrow[from=2-1, to=2-2]
                    \end{tikzcd}
                $$

            Finally, recall that in abelian categories, finite products coincide with finite direct sums.
        \end{remark}
        \begin{lemma}[Tangent spaces of pullbacks] \label{lemma: tangent_spaces_of_pullbacks}
            Suppose that $\varphi: X \to S$ and $\psi: Y \to S$ are two morphisms of $C^k$-manifolds and that the pullback:
                $$
                    \begin{tikzcd}
                	{X \x_{\varphi, S, \psi} Y} & Y \\
                	X & S
                	\arrow[from=1-1, to=1-2]
                	\arrow[from=1-1, to=2-1]
                	\arrow["\lrcorner"{anchor=center, pos=0.125}, draw=none, from=1-1, to=2-2]
                	\arrow[from=1-2, to=2-2]
                	\arrow[from=2-1, to=2-2]
                    \end{tikzcd}
                $$
            exists in $C^k\-\Mfd$. Let $(x, y) \in X \x_{\varphi, S, \psi} Y$ be a point; by the construction of pullbacks, we are guaranteed that $\varphi(x) = \psi(y)$, so let us set $s := \varphi(x) = \psi(y)$. Then, the tangent space $T_{X \x_{\varphi, S, \psi} Y, (x, y)}$ can be formed as the following pullback:
                $$
                    \begin{tikzcd}
                	{T_{X \x_{\varphi, S, \psi} Y, (x, y)}} & {T_{Y, y}} \\
                	{T_{X, x}} & {T_{S, s}}
                	\arrow[from=1-1, to=1-2]
                	\arrow[from=1-1, to=2-1]
                	\arrow["\lrcorner"{anchor=center, pos=0.125}, draw=none, from=1-1, to=2-2]
                	\arrow["{d\psi_y}", from=1-2, to=2-2]
                	\arrow["{d\varphi_x}", from=2-1, to=2-2]
                    \end{tikzcd}
                $$
            or equivalently, we have that $T_{X \x_{\varphi, S, \psi} Y, (x, y)} \cong \ker(d\varphi_x \oplus d\psi_y)$.
        \end{lemma}
            \begin{proof}
                
            \end{proof}
        \begin{corollary}[Base-changing submersive/unramified/\'etale morphisms] \label{coro: base_changing_submersive_unramified_and_etale_morphisms}
            
        \end{corollary}
            \begin{proof}
                
            \end{proof}

        \begin{lemma}
            Consider an ambient $n$-manifold $N$ and two submanifolds $S$ and $M$, such that:
            \begin{itemize}
                \item $S$ is a submanifold of $N$ that is of codimension $1 \leq k < n$, and
                \item $M$ is given by a smooth inclusion $f: M \hookrightarrow N$.
            \end{itemize}
            If $f$ is transverse to the submanifold $S \subset N$, then $f^{-1}(S)$ will be an embedded submanifold of codimension $k = \codim S$.
        \end{lemma}
            \begin{proof}
                Denote the embedding of $S$ into $N$ by $\iota: S \hookrightarrow N$. Because $S$ is embedded, $\iota$ is smooth.
            
                Let $p \in f^{-1}(S)$ and let $(U, \varphi) \ni f(p)$ be a chart for $N$ with local coordinates $(x_1, ..., x_n)$ such that $x_{n - k + 1} = ... = x_n = 0$ on $U \cap S$. Let:
                    $$g: U \to \R^k$$
                be given $g(q) = (x_{n - k + 1}(q), ..., x_n(q))$ for all $q \in U$.

                Let us prove firstly that $f^{-1}(U) \cap f^{-1}(S) = (g \circ f)^{-1}(0)$. By hypothesis, $(U, \varphi) \ni f(p)$ is a chart for $N$ with local coordinates $(x_1, ..., x_n)$ such that $x_{n - k + 1} = ... = x_n = 0$ on $U \cap S$, so:
                    $$g|_{U \cap S}(q) = (x_{n - k + 1}(q), ..., x_n(q)) = (0, ..., 0)$$
                for all $q \in U \cap S$. From this, we see that:
                    $$g^{-1}(0) = U \cap S$$
                and hence:
                    $$(g \circ f)^{-1}(0) = f^{-1}( g^{-1}(0) ) = f^{-1}(U \cap S) = f^{-1}(U) \cap f^{-1}(S)$$

                Now, let us assume that $f$ is transverse to $S$ and then let us prove the main claim. To show that $f^{-1}(S) \subset M$ is an embedded smooth submanifold of codimension $k$, by \cite[Proposition 5.2]{lee_smooth_manifolds}, we must show that for every smooth chart $W \subseteq M$, the intersection $W \cap f^{-1}(S)$ is an embedded submanifold of codimension $k$ inside $W$. Actually, since $f: M \to N$ is a smooth inclusion, \cite[Theorem 4.25]{lee_smooth_manifolds} tells us that it is sufficient to consider $W := f^{-1}(U)$ for some smooth chart $U \subset N$, so we must show that $f^{-1}(U) \cap f^{-1}(S)$ is an embedded smooth submanifold of codimension $k$ inside $f^{-1}(U)$. From the fact that:
                    $$f^{-1}(U) \cap f^{-1}(S) = (g \circ f)^{-1}(0)$$
                we see that we can equivalently show that $(g \circ f)^{-1}(0)$ is an embedded smooth submanifold of codimension $k$ inside $f^{-1}(U)$.

                $f$ and $g$ are both smooth, hence so is the composition $g \circ f$. We also have that $\im g \subseteq \R^k$, so $\im(g \circ f) \subseteq \R^k$ too. Smooth maps are open \textit{a priori}, so $T_qV \cong T_q\R^k$ for all $q \in \im(g \circ f)$ and all open neighbourhood $V \ni q$. By \cite[Corollary 5.14]{lee_smooth_manifolds}, we now need to show that $(g \circ f)^{-1}(0)$ is a regular level set. By the definition of regular level sets, this means showing that 
                    $$d(g \circ f)_p = \left( T_pM \xrightarrow[]{df_p} T_{f(p)}N \xrightarrow[]{dg_{f(p)}} T_{(g \circ f)(p)}\R^k \right)$$
                is surjective for all $p \in (g \circ f)^{-1}(0)$. As a last reduction step, note that $\dim T_{(g \circ f)(p)}\R^k \cong \dim \R^k = k$, so in terms of the local coordinates $(x_1, ..., x_n) \in \varphi(U)$, the Jacobian of $g$ at any point $q \in \im f$ is the following $k \x n$ matrix:
                    $$
                        dg_q =
                        \begin{pmatrix}
                            \frac{\del g(x_1, ..., x_n)}{\del x_1}
                            \\
                            \vdots
                            \\
                            \frac{\del g(x_1, ..., x_n)}{\del x_n}
                        \end{pmatrix}^{\top} \bigg|_{(x_1, ..., x_n) = q}
                        =
                        \begin{pmatrix}
                            \frac{\del (x_{n - k + 1}, ..., x_n)}{\del x_1}
                            \\
                            \vdots
                            \\
                            \frac{\del (x_{n - k + 1}, ..., x_n)}{\del x_n}
                        \end{pmatrix}^{\top} \bigg|_{(x_1, ..., x_n) = q}
                    $$
                wherein the columns are:
                    $$\frac{\del (x_{n - k + 1}, ..., x_n)^{\top}}{\del x_i} = (0, ..., 0, 1, 0, ..., 0)^{\top}$$
                with $1$ in the $i^{th}$ entry and $0$ elsewhere. This means that $g$ is of constant rank $k$ by construction, and hence $dg_q$ is already surjective. Thus, it remains to demonstrate that $df_p: T_pM \to T_{f(p)}N$ is surjective at all points $p \in (g \circ f)^{-1}(0)$.
                
                Let $p \in f^{-1}(S)$ and let $U \ni f(p)$ be a smooth chart of $N$. The map $\iota: S \hookrightarrow N$ is a smooth embedding by hypothesis, hence a smooth immersion in particular, and so $d\iota_{f(p)}: T_{f(p)}S \to T_{f(p)}N$ is injective, i.e. $\ker d\iota_{f(p)} = \{0\}$. This then implies that:
                    $$\ker\left( T_pM \oplus T_{\iota(p)}S \xrightarrow[]{df_p + d\iota_{f(p)}} T_{f(p)}N \right) \cong \ker df_p$$
                By the First Isomorphism Theorem. this implies that:
                    $$\im(df_p + d\iota_{f(p)}) \cong \im df_p$$
                Now, by transversality, we have that:
                    $$T_{f(p)}N = \im df_p + \im d\iota_{f(p)} = \im( df_p + d\iota_{f(p)} )$$
                from which we infer that:
                    $$T_{f(p)}N \cong \im(df_p + d\iota_{f(p)}) \cong \im df_p$$
                This clearly shows that $df_p$ is surjective, so we are done.
            \end{proof}
        \begin{theorem}[Pullbacks that exist: transversal intersections] \label{theorem: transversal_intersections}
            
        \end{theorem}
            \begin{proof}
                
            \end{proof}

        \begin{example}[Smooth structures on spheres] \label{example: smooth_structures_on_spheres}
            By construction, the $n$-sphere $\bbS^n$ is the level set $\Phi^{-1}(1)$ of the smooth function:
                $$\Phi: \R^{n + 1} \to \R$$
            given by:
                $$\Phi(x_1, ..., x_{n + 1}) := \sum_{i = 1}^{n + 1} x_i^2$$
        \end{example}