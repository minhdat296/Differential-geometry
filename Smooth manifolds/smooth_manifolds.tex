\section{The category of topological manifolds vs. the category of smooth manifolds}
    \subsection{Topological manifolds}
        We begin by discussing topological manifolds, which are very special topological spaces which locally resemble $\R^n$ in a precise sense. For readers familiar with ringed spaces, these topological manifolds will be the underlying spaces of the smooth manifolds that we will discuss in the next subsection: to us, smooth manifolds will be topological manifolds equipped with $C^{\infty}$ structure sheaves of $\R$-algebras, so-called \say{smooth structures}.
    
        \begin{definition}[Krull dimension] \label{def: krull_dimension}
            (Cf. \cite[\href{https://stacks.math.columbia.edu/tag/0055}{Tag 0055}]{stacks}) Let $X$ be a topological space. Its \textbf{Krull dimension} is the element of $\N \cup \{+\infty\}$ given by:
                $$\dim X := \sup \{ \text{lengths of finite chains $Z_0 \subset Z_1 \subset ... \subset Z_n \subset X$, where each $Z_i$ is irreducible} \}$$
            Furthermore, we take:
                $$\dim \varnothing := 0$$
        \end{definition}
        \begin{definition}[Topological manifolds] \label{def: topological_manifolds}
            For us, the word \textbf{topological manifold} of dimension $n$ (for some $n \in \N$) (or \textbf{manifold} of dimension $n$ or \textbf{$n$-manifold}, for short) shall mean a topological space $X$ that:
            \begin{itemize}
                \item dimension $n$,
                \item is Hausdorff,
                \item admits a countable basis for its topology (i.e. \say{second-countable}), and
                \item is locally homeomorphic to some open subset of $\R^n$ with the usual Euclidean metric topology (i.e. \say{locally Euclidean}), which is to say that for each point $x \in X$, there exists an open neighbourhood $U \ni x$ and an open subset $U' \subseteq \R^n$ for which there is a homeomorphism $U \cong U'$; we note also that $n$ remains constant as $x \in X$ varies.
            \end{itemize}
            
            Sometimes, people also say \say{real manifolds}, but in our opinion, this is only a useful terminology for differentiating between \say{smooth real manifolds} (to be defined) and so-called \say{holomorphic complex manifolds}. Otherwise, the term can cause unnecessary confusion.
        \end{definition}
        \begin{remark}
            An immediate consequence of the last condition is that the connected components of a manifold of dimension $n$ are all equally of Krull dimension $n$.
        \end{remark}
        
        Of course, since the Euclidean topology on $\R^n$ is generated by open $n$-balls, it is sufficient to only require that points of a manifold lie inside open neighbourhoods that are homeomorphic to some open $n$-ball.
        \begin{convention}
            If $x_0 \in \R^n$ is a point that we designate to be a centre and if we choose some $\e > 0$, then we will write:
                $$\B^n_\e(x_0) := \{ x \in \R^n \mid \| x - x_0 \| < \e \}$$
            for the open $n$-ball of radius $r$ centered at $x_0$. Again, it is well-known that such balls form a basis for the standard Euclidean topology on $\R^n$.

            Sometimes, we will also need closed $n$-discs/balls, which shall be given by:
                $$\bbD^n_\e(x_0) := \{ x \in \R^n \mid \| x - x_0 \| \leq \e \}$$
        \end{convention}
        \begin{remark}[Some basic topological properties of manifolds]
            Definition \ref{def: topological_manifolds} also hints at the following topological properties of manifolds. We leave proofs to the reader as exercises. For what follows, let $X$ be a manifold.
            \begin{enumerate}
                \item \textbf{(Path-connectedness):} Any connected component of $X$ is path-connected. Hence, a manifold is path-connected if and only if it is connected, and any manifold is locally path-connected.
                \item \textbf{(Countably many connected components):} Each connected component of $X$ is open, and hence the collection of connected components form an open cover of $X$. Due to second-countability, this also implies that any manifold has countably many connected components.
                \item \textbf{(Para-compactness):} Any manifold is locally compact. Along with second-countability, this implies that every manifold is para-compact (every open cover has a locally finite refinement).
                \item \textbf{(Fundamental groups):} $\pi_1(X)$ is countable.
            \end{enumerate}
        \end{remark}

        Let us also remark right away that while it may seem intuitively true that if $m \not = n$ then a manifold of dimension $m$ can not be homeomorphic to one that is of dimension $n$, but there are some subtleties. We will return to this question later.

        \begin{definition}[Atlases and coordinate charts] \label{def: atlases_and_charts}
            Let $X$ be a manifold of dimension $n$ and choose a covering $X := \bigcup_{i \in I} U_i$ by open subsets $U_i$. An \textbf{atlas} for $X$ is then a family of homeomorphisms $\{ \varphi_i: U_i \xrightarrow[]{\cong} \R^n \}_{i \in I}$. Elements of an atlas are called \textbf{coordinate charts} and are usually denoted, e.g. by $(U_i, \varphi_i)$.
        \end{definition}
        \begin{example}[Graphs of continuous functions] \label{example: graph_coordinates}
            Let $X$ be an $n$-manifold and $U$ be an open subset thereof. If $f: U \to \R^m$ is a continuous function then let us identify its graph with the subspace $U \x \im f$ of $\R^{n + m}$. This gives rise to an open immersion:
                $$\Gamma_f: U \x \im f \to \R^n \x \R^m$$
            such that:
                $$\pr_1 \circ \Gamma_f = f \circ \pr_1|_{U \x \im f}$$
            Then, note that $\pr_1 \circ \Gamma_f$ is in fact a homeomorphism with continuous inverse given by $( \pr_1 \circ \Gamma_f )^{-1}(x) := ( x, f(x) )$ for all $x \in U$. As such, we have found a homeomorphism:
                $$\varphi \circ (\pr_1 \circ \Gamma_f): U \x \im f \xrightarrow[]{\cong} \R^n$$
            where on the RHS we have made an identification $\varphi: U \xrightarrow[]{\cong} \R^n$ (which exists, as $U$ is an open subset of an $n$-manifold), and hence have found a chart $(U \x \im f, \varphi)$.
        \end{example}
        \begin{example}[Spheres]
            An instance of the kind of coordinate charts discussed in example \ref{example: graph_coordinates} is the following.

            Let $X := \bbS^n_{\R}$ be the real unit $n$-sphere, given by:
                $$\bbS^n_{\R} := \{ z \in \R^{n + 1} \mid \|z\| = 1 \}$$
            and as such is a closed subset of $\R^{n + 1}$, and hence is Hausdorff and second-countable. We wish to show that $\bbS^n_{\R}$ is actually an $n$-manifold, and the only thing left to check is that it is locally homeomorphic to $\R^n$.
            
            To this end, consider the function:
                $$f: \bbS^n_{\R} \to \R$$
            given locally around each point $z \in \bbS^n_{\R}$ by:
                $$f(x) := \sqrt{1 - \|x\|^2}$$
            for all $x \in \B^n_1(z)$. This function is clearly continuous, and thus gives rise to open immersions:
                $$\Gamma_{f, z}: \B^n_1(z) \x \im f \to \R^n \x \R$$
            for each $z \in \bbS^n_{\R}$, which in turn gives rise to homeomorphisms:
                $$\varphi_z \circ (\pr_1 \circ \Gamma_{f, z}): \B^n_1(z) \x \im f \xrightarrow[]{\cong} \R^n$$
            wheren we have identified $\varphi_z: \B^n_1(z) \xrightarrow[]{\cong} \R^n$ (cf. example \ref{example: graph_coordinates}). Thus, we have shown that $\bbS^n_{\R}$ is locally isomorphic to $\R^n$, and hence an $n$-manifold.
        \end{example}

        \textit{We insist on the point of view that topology and geometry are best understood structurally, which means that every construction should form some kind of a category, and we will pursue this viewpoint by investigating the kinds of limits and colimits that can exist in these categories.} Manifolds are no exception. 
        \begin{proposition}[The category of topological manifolds]
            \begin{enumerate}
                \item There is a category $\Mfd$ whose objects are manifolds (of all dimensions $n \in \N$) and whose morphisms are continuous maps between them.
                \item This category is closed under open immersions and finite pullbacks. Furthermore, if $X$ is an $n$-manifold and $U \hookrightarrow  X$ is an open immersion, then $\dim U = \dim X$, and the pullback of two open immersions, say $U \hookrightarrow  X, U' \hookrightarrow  X$ coincides with the intersection $U \cap U'$ (with the canonically induced open immersion $U \cap U' \to X$). 
                \item $\Mfd$ is also closed under locally closed immersions, which is to say that if $X$ is a manifold and $Z \hookrightarrow X$ is a locally closed immersion then $Z$ will also be a manifold. However, if this locally closed immersion is not a homeomorphism, then $\dim Z < \dim X$ strictly.
            \end{enumerate}
        \end{proposition}
            \begin{proof}
                \begin{enumerate}
                    \item To check that $\Mfd$ is a well-defined category, the only non-trivial thing to check is whether given manifolds $X, Y$ and a continuous map $f: X \to Y$, as well as an arbitrary open subset $V \subseteq Y$ that is homeomorphic to some $\R^m$, then its preimage $f^{-1}(V)$ will also be homeomorphic to some $\R^n$. As stated above, we can assume without any loss of generality that $V$ is homeomorphic to some open ball inside $\R^m$. Then, its preimage $f^{-1}(V)$ will be homeomorphic to some open ball inside $\R^n$, per the continuity of $f$. Any open ball in $\R^n$ is homeomorphic to $\R^n$ itself (this is well-known), so we are done.
                    \item These statements follow directly from the definition of manifolds.
                    \item Manifolds are locally compact and Hausdorff, and hence locally closed (by the Bolzano-Weierstra{\ss} Theorem). From the previous statement, we also know that if $X$ is an $n$-manifold and $U \hookrightarrow  X$ is an open immersion, then $\dim U = \dim X$. By putting the two statements together, and by assuming without any loss of generality that $X \cong \R^n$, then the assertion will following from the fact that any locally closed subset $Z \subseteq \R^n$ is an $m$-manifold for some $m \leq n$ (as they are automatically Hausdorff, second-countable, and one can prove that they are locally homeomorphic to $\R^m$ by picking charts), and that equality occurs if and only if $Z \not \cong \R^n$.
                \end{enumerate}
            \end{proof}

    \subsection{Smooth structures}
        Since manifolds locally resemble Euclidean spaces, it makes sense to speak of derivatives of maps between manifolds, which is a purely local construction.
        \begin{definition}[$C^k$-functions] \label{def: C_k_functions}
            Let $X, Y$ be manifolds and $f: X \to Y$ be a morphism. Such a map is said to be $C^k$ if and only if it admits \textit{continuous} $k^{th}$ order partial derivatives, and if it admits continuous partial derivatives of all orders, then we shall say that it is $C^{\infty}$ or \textbf{smooth}. $C^0$ means continuity.

            The set of $C^k$-functions from $X$ to $Y$ is denoted by $C^k(X, Y)$, and note that when $Y = \R$, this is furthermore a commutative $\R$-algebra via pointwise addition and multiplication.
        \end{definition}
        We remind the reader that one takes the derivative of $f$ at a point $x \in X$ by firstly choosing an open neighbourhood $V \ni f(x)$ along with charts $(V, \psi), (f^{-1}(V), \varphi)$, which identify $V$ and its preimage $f^{-1}(V)$ with, say $\R^m$ and $\R^n$. The rest is vector calculus.

        In light of definition \ref{def: C_k_functions}, let us consider the following. First, suppose that $X$ is a topological manifold and let $X_{\open}$ denote the small\footnote{This site is small because a manifold has only countably many connected component and is locally compact.} site whose objects are open immersions $U \hookrightarrow X$ and whose morphisms are the commutative triangles in $\Mfd$ between such maps. Next, choose some $k \in \N \cup \{+\infty\}$ consider the following distinguished object of $\Psh(X_{\open})$:
            $$\scrO_X$$
        which is given by:
            $$\scrO_X(U) := C^k(U, \R)$$
        for each $U \in \Ob(X_{\open})$. Note that $\scrO_X$ is indeed a well-defined prsheaf on $X_{\open}$: given open subsets $U, U' \subseteq X$ and a continuous function $\varphi_1: U_1 \to U_2$ compatible with those open immersions, then there will be an $\R$-algebra homomorphism:
            $$\scrO_X(\varphi_1): C^k(U_2, \R) \to C^k(U_1, \R)$$
        (note the contravariance!) given by:
            $$\scrO_X(\varphi_1)(g) := g \circ \varphi$$
        and given two more open subsets $U_2, U_3,  \subseteq X$ and continuous functions $\psi_i: U_i \to U_{i + 1}$, then we will get that:
            $$\scrO_X(\varphi_3 \circ \varphi_2 \circ \varphi_1)(g) := g \circ ( \varphi_3 \circ \varphi_2 \circ \varphi_1 )$$
        from which one easily infers associative. Now, we claim that $\scrO_X$ is actually a sheaf. For convenience, let us for now also say that the pair:
            $$(X, \scrO_X)$$
        as above is a \textbf{$C^k$-premanifold} and we will be referring to $\scrO_X$ as the \textbf{structure presheaf}.
        \begin{proposition}[$C^k$-premanifolds are ringed spaces] \label{prop: premanifolds_are_ringed_spaces}
            For any $C^k$-premanifold $(X, \scrO_X)$, the structure presheaf $\scrO_X$ satisfies descent, i.e. is a sheaf on $X_{\open}$. 
        \end{proposition}
            \begin{proof}
                
            \end{proof}

        \begin{definition}[$C^k$-manifolds] \label{def: C_k_manifolds}
            For a given $k \in \N \cup \{+\infty\}$, a \textbf{$C^k$-manifold} is a manifold $X$ equipped with the sheaf of $\R$-algebras $\scrO_X$ on $X_{\open}$ that is given by:
                $$\scrO_X(U) := C^k(U, \R)$$
            A $C^{\infty}$-manifold is also commonly called a \textbf{smooth manifold}.
        \end{definition}
        
        \begin{proposition}
            For each $k \in \N \cup \{+\infty\}$, there exists a category $C^k\-\Mfd$ of $C^k$-manifolds and $C^k$-maps between them as morphisms. Note that $C^0\-\Mfd = \Mfd$, tautologically.
        \end{proposition}
            \begin{proof}
                
            \end{proof}

        \begin{proposition}[The $C^k$-topologies] \label{prop: C_k_topologies}
            
        \end{proposition}
            \begin{proof}
                
            \end{proof}
        \begin{lemma}[$C^k$-structures] \label{lemma: C_k_structures}
            Let $(X, \scrO_X)$ be a $C^k$-manifold, for some $k \in \N \cup \{+\infty\}$. Then, the structure sheaf $\scrO_X$ will also satisfy $C^k$-descent.
        \end{lemma}
            \begin{proof}
                
            \end{proof}
    
        \todo[inline]{Existence of partitions of unity implies acyclicity of structure sheaves}

    \subsection{Submersions, transversality of intersections, and fibred products}